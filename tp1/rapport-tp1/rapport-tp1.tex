\documentclass{rapport}
\usepackage[utf8]{inputenc}

\usepackage{pifont} % Pour les symboles appelés par la macro \ding
\usepackage{url} % Comme son nom l'indique, pour les url...

\usetikzlibrary{positioning} % Bibliothèque tikz pour positionner des nœuds relativement à d'autres

\usepackage[colorlinks, citecolor=red!60!green, linkcolor=blue!60!green, urlcolor=magenta]{hyperref} % Pour que les liens soient cliquables. Les options permettent de mettre les liens en couleur.

\usepackage{algorithm}
\usepackage{algo}
\usepackage{colorationSyntaxique}
\usepackage{amsmath}
\usepackage{diagbox}

% Pour un rapport en français 
\usepackage[français]{babel} % Commenter pour un rapport en anglais
\renewcommand\bibsection{\section*{Bibliographie}} % Commenter pour un rapport en anglais

% \englishTitlePage % Décommenter pour une page de titre en anglais

\usepackage{listings}
\lstset{                  % Specify language
    basicstyle=\ttfamily\small,     % Code font and size
    keywordstyle=\color{blue},      % Color for keywords
    commentstyle=\color{gray},      % Color for comments
    stringstyle=\color{red},        % Color for strings
    numbers=left,                   % Add line numbers
    numberstyle=\tiny\color{gray},  % Style for line numbers
    % frame=single,                   % Add a border around code
    breaklines=true,                % Line wrapping
    % backgroundcolor=\color{gray!10} % Light gray background
}

\pagestyle{fancy}
\renewcommand{\sectionmark}[1]{\markboth{\thesection.\ #1}{}}
\fancyfoot{}

\fancyhead[LE]{\textsl{\leftmark}}
\fancyhead[RE, LO]{\textbf{\thepage}}
\fancyhead[RO]{\textsl{\rightmark}}

\def\Latex{\LaTeX\xspace}
\def\etc{\textit{etc.}\xspace}



\title{Évaluation des performances d’un programme et optimisations de code par compilation}
\author{Mehdi Mansour}
\supervisor{Professeur Sid Touati}
\date{Premier semestre de l'année 2024-2025}

%\universityname{Université Côte d'Azur} % Nom de l'université.
\type{TP1} % Type de document
% \formation{Master Informatique} % Nom de la formation

% Retrouver les autres options possibles dans le document rapport.pdf

\begin{document}

  \maketitle

  \begin{abstract}
    Ce rapport a pour objectif de détailler les différents résultats obtenus lors de ce TP1 disponible ci-après : \url{https://lms.univ-cotedazur.fr/2024/pluginfile.php/228842/mod_resource/content/10/Eval-Perf.pdf}.
     \end{abstract}
    %\newcommand{\ppc}{programmation par contraintes\xspace}
\subsection*{Architecture utilisée pour ce TP}
     \noindent
    CPU name:	Intel(R) Core(TM) i7-10700F CPU @ 2.90GHz
    \newline
    CPU type:	Intel Cometlake processor
    \newline
    CPU stepping:	5
    \newline
    \newline
    \noindent
    Hardware Thread Topology
    \newline
    \newline
    Sockets:		1
    \newline
    Cores per socket:	8
    \newline
    Threads per core:	2
    \newline
    Socket 0:		( 0 8 1 9 2 10 3 11 4 12 5 13 6 14 7 15 )
    \newline
    \newline
    NUMA domains:		1
    \newline
    Domain:			0
    \newline
    Processors:		( 0 8 1 9 2 10 3 11 4 12 5 13 6 14 7 15 )
    \newline
    Distances:		10
    \newline
    Free memory:		21746.9 MB
    \newline
    Total memory:		31991.4 MB
    \newline

\subsection*{Topologie du PC}
    \begin{table}[h!]
    \centering
    \begin{tabular}{|c|c|c|c|c|c|c|c|c|}
        \hline
        \multicolumn{9}{|c|}{Topologie graphique} \\
        \hline
        Core & \enspace0\enspace\enspace8 &\enspace1\enspace\enspace9 &\enspace2\enspace\enspace10 &\enspace3\enspace\enspace11 &\enspace4\enspace\enspace12 &\enspace5\enspace\enspace13 &\enspace6\enspace\enspace14 &\enspace7\enspace\enspace15\\
        \hline
        Cache L1& \enspace32 kB &\enspace32 kB &\enspace32 kB &\enspace32 kB &\enspace32 kB&\enspace32 kB&\enspace32 kB&\enspace32 kB\\
        \hline
        Cache L2 & 256 kB & 256 kB & 256 kB & 256 kB & 256 kB& 256 kB& 256 kB& 256 kB\\
        \hline
        Cache L3 & \multicolumn{8}{|c|}{16 MB} \\
        \hline
    \end{tabular}
    %\caption{Les différents niveaux de cache par cœur}
    \label{tab:graph_characteristics}
    \end{table}
  
  \clearpage
  \tableofcontents

  \clearpage
%«»

  \part{Introduction}
    L'objectif de ce TP est d'analyser les résultats d'exécution de deux fichiers C (disponibles dans l'archive fournisse) pour évaluer leurs performances.
    Ces fichiers «.c», nommés «dynamique.c» et «statique.c» sont composés de plusieurs fonctions décrites dans le sujet du TP : 
    \begin{itemize}
        \item Addition de deux vecteurs de valeurs flottantes C = A + B
    \end{itemize}
    \begin{itemize}
        \item Multiplication de deux matrices C = A × B
    \end{itemize}
    \begin{itemize}
        \item Multiplication d’un vecteur par un scalaire A = A × s
    \end{itemize}
    Lors de ce TP nous utiliserons plusieurs outils comme la commande \textit{time} ou \textit{gprof} pour analyser notre code, ainsi que des options d'optimisation pour améliorer sa rapidité d'exécution.
    \newline
    \newline
    \indent Ces analyses sont nécessaires pour optimiser les bouts de code qui prennent beaucoup de temps. Nous pouvons lors de la compilation réduire ce temps de calcul grâce à des options d'optimisation que nous développerons plus tard (voir man gcc).     

  %\pageblanche
  \clearpage
  \part{Réalisation des benchmarks}\setcounter{section}{0}
  
    Dans cette partie nous réaliserons des benchmarks pour avoir un ordre d'idée du temps de calcul de notre programme.
    Un benchmark est une série de tests réalisés pour analyser les performances d'un appareil informatique, composants (ex : carte graphique) ou dans notre cas d'un programme.
    
    \section{Détail des fonctions}
      Dans cette partie nous allons détailler les fonctions décrites dans l'introduction. Nous utiliserons des matrices et vecteurs avec des données déclarées statiquement comme globales et des données allouées dynamiquement avec \textit{malloc}. Les données sont des valeurs flottantes. Dans chaque fonction, on print dans un fichier une valeur pour éviter les codes morts et faire en sorte que le compilateur exécute exactement ce que l'on veut.
      \subsection{Addition de deux vecteurs de valeurs flottantes C = A + B}
        Voici la fonction d'addition de deux vecteurs.
        \begin{algorithm}
        \begin{C}
void add() {
	freopen("error.log", "w", stderr);
    for (int i = 0; i < N; i++) {
        C1[i]=A1[i]+B1[i];
        fprintf(stderr,"%f",C1[i]);
    }
    fclose(stderr);
}
        \end{C}
        \caption[Algo en C]{: Addition de deux vecteurs\label{ag:algoc}}
        \end{algorithm}
        
      \subsection{Multiplication de deux matrices C = A × B}
        Voici la fonction qui fait la multiplication de deux matrices.
        \begin{algorithm}
        \begin{C}
void multMatrice(){
    freopen("error.log", "w", stderr);
    for(int i=0; i< N; i++){
        for (int j=0; j< M; j++){
            for (int k=0; k< P; k++){
                C[i][j] = C[i][j] + A[i][k] * B[k][j];
            }
            fprintf(stderr,"%f",C[i][j]);
        }
    }
    fclose(stderr);
}
        \end{C}
        \caption[Algo en C]{Retourne la valeur maximale du tableau tab.\label{ag:algoc}}
        \end{algorithm}

      \subsection{Multiplication d’un vecteur par un scalaire A = A × s}
        Voici la fonction de multiplication d'un vecteur par un scalaire.
        \begin{algorithm}
        \begin{C}
void mult_scalaire(){
    freopen("error.log", "w", stderr);
    for(int i=0;i<N;i++){
        As[i]=As[i]*s;
        fprintf(stderr,"%f",As[i]);
    }
    fclose(stderr);
}
        \end{C}
        \caption[Algo en C]{Retourne la valeur maximale du tableau tab.\label{ag:algoc}}
      \end{algorithm}
    
    \section{Mesure des performances et profilage d’un programme}
    \subsection{Temps d'exécution}
      Pour évaluer le temps d'exécution de notre programme, nous utiliserons la commande \textit{time}.
      Pour plus d'efficacité nous exécuterons un makefile pour effectuer nos commandes.
      
    \subsubsection{fonction addition}
      Pour la fonction addition, on obtient ces résultats (exécutés avec la ligne de commande \textit{make test\_add}, voir résultat dans \textit{resultat/res\_test\_add.txt}) :
     
    \begin{table}[h!]
    \centering
    \begin{tabular}{|c|c|c|c|}
        \hline
        Version & Temps Total & Temps Utilisateur & Temps Système \\
        \hline
        Statique & 0,548s & 0,400s & 0,148s \\
        \hline
        Dynamique & 0,454s & 0,281s & 0,173s \\
        \hline
    \end{tabular}
    \caption{Comparaison entre programme statique et dynamique (Exécution 1)}
    %\label{tab:graph_characteristics_1}
\end{table}

\begin{table}[h!]
    \centering
    \begin{tabular}{|c|c|c|c|}
        \hline
        Version & Temps Total & Temps Utilisateur & Temps Système \\
        \hline
        Statique & 0,562s & 0,426s & 0,136s \\
        \hline
        Dynamique & 0,441s & 0,297s & 0,144s \\
        \hline
    \end{tabular}
    \caption{Comparaison entre programme statique et dynamique (Exécution 2)}
    %\label{tab:graph_characteristics_2}
\end{table}
  
  On voit une légère différence entre les données statiques et celles allouées dynamiquement avec \textit{malloc}.
  \begin{itemize}
        \item le temps real représente le temps total effectué (en seconde).
    \end{itemize}
    \begin{itemize}
        \item le temps user représente le temps CPU utilisateur (exécution des instructions du programme)
    \end{itemize}
    \begin{itemize}
        \item le temps sys représente le temps CPU système (appel système, opération entrées/sorties ...)
    \end{itemize}\textit{}
    On a pour la première exécution, avec données statiques, un temps total de 0,548s secondes. Le temps utilisateur représente ici environ 73\% du temps total ((0,400/0,548)*100=72.9). Le temps système quant à lui représente environ 27\%.
    62\% environ pour le temps utilisateur avec données allouées dynamiquement.
    \newline
  \indent On remarque également que pour chaque exécution, on a des résultats différents et pour une raison simple : il y'a d'autres processus qui s'exécutent en même temps que notre programme, ce qui explique ces différences. Par exemple pour la deuxième exécution, le temps utilisateur de la version statique représente 75\% du temps total, soit 2 points de différence.
 \subsubsection{Fonction multiplication matrice}
  Pour la fonction multiplication on obtient ces résultats (exécutés avec la ligne de commande \textit{make test\_mult}, voir résultat dans \textit{resultat/res\_test\_mult.txt}):

      \begin{table}[h!]
    \centering
    \begin{tabular}{|c|c|c|c|}
        \hline
        Version & Temps Total & Temps Utilisateur & Temps Système \\
        \hline
        Statique & 18,018s & 17,749s & 0,240s \\
        \hline
        Dynamique & 33,561s & 33,334s & 0,204s \\
        \hline
    \end{tabular}
    \caption{Comparaison entre programme statique et dynamique (Exécution 1)}
    \label{tab:graph_characteristics_1}
\end{table}

\begin{table}[h!]
    \centering
    \begin{tabular}{|c|c|c|c|}
        \hline
        Version & Temps Total & Temps Utilisateur & Temps Système \\
        \hline
        Statique & 17,303s & 17,007s & 0,192s \\
        \hline
        Dynamique & 28,881s & 28,570s & 0,232s \\
        \hline
    \end{tabular}
    \caption{Comparaison entre programme statique et dynamique (Exécution 2)}
    %\label{tab:graph_characteristics_2}
\end{table}

  Pour la version statique, on a un temps utilisateur qui représente 98\% du temps total contre 2\% pour le temps système. 
  Ces résultats nous montrent que l'exécution de la fonction multiplication de matrices est très longue. On remarque également que le temps d'exécution est différent d'une exécution à une autre.
\subsubsection{fonction scalaire}
  Pour la fonction multiplication d'un vecteur par un scalaire on obtient ce résultat (exécuté avec la ligne de commande \textit{make test\_scal}):
  
      \begin{table}[h!]
    \centering
    \begin{tabular}{|c|c|c|c|}
        \hline
        Version & Temps Total & Temps Utilisateur & Temps Système \\
        \hline
        Statique &0,471s & 0,327s & 0,144s \\
        \hline
        Dynamique & 0,425s & 0,284s & 0,140s \\
        \hline
    \end{tabular}
    \caption{Comparaison entre programme statique et dynamique (Exécution 1)}
    %\label{tab:graph_characteristics_1}
\end{table}

\begin{table}[h!]
    \centering
    \begin{tabular}{|c|c|c|c|}
        \hline
        Version & Temps Total & Temps Utilisateur & Temps Système \\
        \hline
        Statique & 0,480s & 0,344s & 0,136s \\
        \hline
        Dynamique & 0,418s & 0,287s & 0,131s \\
        \hline
    \end{tabular}
    \caption{Comparaison entre programme statique et dynamique (Exécution 2)}
    %\label{tab:graph_characteristics_2}
\end{table}

      Pour la première exécution version statique, le temps utilisateur représente 69\%, le temps système 31\% du temps total.
      Grâce à ces benchmarks on sait que notre programme va prendre beaucoup de temps à faire la multiplication de matrices. On sait donc quelle partie du code optimiser.
    \subsection{Calcul des métriques de performances}
    Voici ci-dessous les formules des métriques suivantes : CPI, IPC et GFLOPS
\[
\text{CPI(P,I,A)} = \frac{\text{TempsExecution}(P, I, A)\times \text{FrequenceHorloge}(A)}{\text{NombreInstructions}(P, I, A)} 
\]
\[
\text{IPC(P,I,A)} = \frac{1}{\text{CPI(P,I,A)}}
\]
\[
\text{GFLOP} = \frac{\text{Nombre total d'opérations flottantes}}{\text{Temps d'exécution (en secondes)} \times 10^9}
\]
\subsection{Profilage d'un programme}
Pour cet exercice, nous allons profiler avec l'outil \textit{gprof}, le fichier «dynamique.c» avec l'option \textit{-pg} pour permettre à \textit{gprof} de récupérer les données qu'il souhaite. Voici la suite de commande que l'on exécute dans le terminal (\textit{make gprof}, résultat dans \textit{resultat/res.txt}): 
 \begin{lstlisting}
gcc -pg -O0 dynamique.c -o dynamique_gprof
./dynamique_gprof
gprof -b dynamique\_gprof gmon.out > resultat/res.txt
  \end{lstlisting} 
  Voici ci-dessous un extrait de \textit{res.txt} :
  \begin{lstlisting}
time   seconds   seconds    calls   s/call   s/call  name    
 98.33     23.68    23.68        1    23.68    23.68  multMatrice
  1.66     24.08     0.40        1     0.40     0.40  initialiserMatrices
  0.00     24.08     0.00        1     0.00     0.00  add
  0.00     24.08     0.00        1     0.00     0.00  mult_scalaire

  granularity: each sample hit covers 2 byte(s) for 0.04% of 24.08 seconds
  \end{lstlisting} 
%«»
En regardant la colonne «calls», on remarque que 4 fonctions ont été exécutées, celle de la colonne «name» classée dans l'ordre décroissant du temps d'exécution. Ce qui veut dire que «multMatrice» met 23.68s sur 24.08s pour s'exécuter soit 98,33\% du temps.
Les fonctions «add» et «mult\_scalaire» quant à elles, sont à 0 car imperceptible par l'outil. Leur temps d'exécution est très faible.
\clearpage
    \part{Optimisation de code}\setcounter{section}{0}
    \section{Optimisation de code avec le compilateur gcc et icx}
Pour cet exercice, nous allons exécuter un script (ici makefile) pour effectuer nos calculs efficacement, l'objectif étant de faire une comparaison entre les 4 options d'optimisation (-O0, -O1, -O2, -O3, -Os) et les compilateurs gcc et icx. Les résultats obtenus sont redirigés dans des fichiers disponibles dans le dossier \textit{resultat/}.
\subsection{res\_time\_statique.txt et res\_time\_dynamique.txt}
\label{sec:mkl}
%Pour les fichiers res\_time\_statique et res\_time\_tp1 : 
Ces deux fichiers se présentent comme suit : en premier, les 4 exécutions, pour les 4 options d'optimisation différentes, avec le compilateur gcc puis 4 exécutions avec le compilateur icx et enfin une exécution avec l'option \textit{-mkl} d'icx. Les 3 fonctions sont exécutées ensemble.
\subsubsection{res\_time\_statique.txt}
Voici un exemple des résultats obtenus décrits dans res\_time\_statique.txt:
\begin{table}[h!]
    \centering
    \begin{tabular}{|c|c|c|c|c|c|c|}
        \hline
        \diagbox{compilateur}{Temps total} & -O0 & -O1 & -O2 & -O3 & -Os & -mkl\\
        \hline
        gcc & 18,981s & 6,953s & 6,346s & 6,407s & 6,430s & \diagbox{}{} \\
        \hline
        icx & 14,873s & 6,800s &  3,627s & 3,515s & 6,043s & 3,417s\\
        \hline
        taux accélération gcc & 1 & 2,73 & 2.99 & 2,96 & 2,95 & \diagbox{}{} \\
        \hline
        taux accélération icx & 1 & 2,19 & 4,1 & 4,23 & 2,46 & 4,35 \\
        \hline
    \end{tabular}
    \caption{Comparaison entre gcc et icx pour différentes options d'optimisation, version statique}
\end{table}

\subsubsection{res\_time\_dynamique.txt}
Voici un exemple des résultats obtenus décrits dans res\_time\_dynamique.txt:
\begin{table}[h!]
    \centering
    \begin{tabular}{|c|c|c|c|c|c|c|}
        \hline
        \diagbox{compilateur}{Temps total} & -O0 & -O1 & -O2 & -O3 & -Os & -mkl\\
        \hline
        gcc & 28,914s & 14,813s & 8,845s & 8,925s & 11,941s & \diagbox{}{}\\
        \hline
        icx & 27,305s & 9,197s &  9,924s & 10,012s & 9,614s & 9,330s\\
        \hline
        taux accélération gcc & 1 & 1,95 & 3,27 & 3,24 & 2,42 & \diagbox{}{} \\
        \hline
        taux accélération icx & 1 & 2,97 & 2,75 & 2,72 & 2,84 & 2,93\\
        \hline
    \end{tabular}
    \caption{Comparaison entre gcc et icx pour différentes options d'optimisation, version dynamique}
\end{table}

\subsection{res\_time\_add.txt, res\_time\_mult.txt et res\_time\_scal.txt}
Les fichiers se présentent comme suit : dans un premier temps, les exécution statiques avec les 4 exécutions, pour les 4 options d'optimisation différentes, le compilateur gcc, puis 4 exécutions avec le compilateur icx. Dans un second temps, les exécutions dynamiques dans le même ordre que les statiques.

\subsubsection{res\_time\_add.txt}
Voici un exemple des résultats obtenus décrits dans «res\_time\_add.txt»:
\begin{table}[h!]
    \centering
    \begin{tabular}{|c|c|c|c|c|c|}
        \hline
        \diagbox{compilateur}{Temps total} & -O0 & -O1 & -O2 & -O3 & -Os\\
        \hline
        gcc & 0,553s & 0,397s & 0,272s & 0,301s & 0,332s \\
        \hline
        icx & 0,486s & 0,330s &  0,267s & 0,264s & 0,342s\\
        \hline
        taux accélération gcc & 1 & 1,4 & 2,034 & 1,840 & 1,667 \\
        \hline
        taux accélération icx & 1 & 1,473 & 1,818 & 1,838 & 1,419\\
        \hline
    \end{tabular}
    \caption{Comparaison entre gcc et icx pour différentes options d'optimisation, version statique}
\end{table}

\begin{table}[h!]
    \centering
    \begin{tabular}{|c|c|c|c|c|c|}
        \hline
        \diagbox{compilateur}{Temps total} & -O0 & -O1 & -O2 & -O3 & -Os\\
        \hline
        gcc & 0,467s & 0,406s & 0,361s & 0,235s & 0,358s \\
        \hline
        icx & 0,479s & 0,412s &  0,220s & 0,223s & 0,357s\\
        \hline
        taux accélération gcc & 1 & 1,150 & 1,294 & 1,987 & 1,305 \\
        \hline
        taux accélération icx & 1 & 1,163 & 2,173 & 2,148 & 1,344\\
        \hline
    \end{tabular}
    \caption{Comparaison entre gcc et icx pour différentes options d'optimisation, version dynamique}
\end{table}

\subsubsection{res\_time\_mult.txt}
Voici un exemple des résultats obtenus décrit dans «res\_time\_mult.txt»:
\begin{table}[H]
    \centering
    \begin{tabular}{|c|c|c|c|c|c|}
        \hline
        \diagbox{compilateur}{Temps total} & -O0 & -O1 & -O2 & -O3 & -Os\\
        \hline
        gcc & 24,695s & 7,677s & 7,206s & 8,028s & 7,575s \\
        \hline
        icx & 19,513s & 6,747s & 4,068s  & 4,207s & 7,302s\\
        \hline
        taux accélération gcc & 1 & 3,22 & 3,42 & 3,08 & 3,26 \\
        \hline
        taux accélération icx & 1 & 2,89 & 4,79 & 4,64 & 2,67\\
        \hline
    \end{tabular}
    \caption{Comparaison entre gcc et icx pour différentes options d'optimisation, version statique}
\end{table}

\begin{table}[H]
    \centering
    \begin{tabular}{|c|c|c|c|c|c|}
        \hline
        \diagbox{compilateur}{Temps total} & -O0 & -O1 & -O2 & -O3 & -Os\\
        \hline
        gcc & 37,611s & 21,004s & 10,452s & 10,214s & 13,864s \\
        \hline
        icx & 34,989s & 10,889s & 10,294s  & 10,831s & 10,744s \\
        \hline
        taux accélération gcc & 1 & 1,79 & 3,60 & 3,68 & 2,71 \\
        \hline
        taux accélération icx & 1 & 3,22 & 3,40 & 3,23 & 3,26\\
        \hline
    \end{tabular}
    \caption{Comparaison entre gcc et icx pour différentes options d'optimisation, version dynamique}
\end{table}

\subsubsection{res\_time\_scal.txt}
Voici un exemple des résultats obtenus décrit dans «res\_time\_scal.txt»:%«»
    \begin{table}[H]
    \centering
    \begin{tabular}{|c|c|c|c|c|c|}
        \hline
        \diagbox{compilateur}{Temps total} & -O0 & -O1 & -O2 & -O3 & -Os\\
        \hline
        gcc & 0,515s & 0,411s &0,270s  & 0,299s & 0,328s \\
        \hline
        icx & 0,485s & 0,339s & 0,254s  & 0,298s & 0,324s\\
        \hline
        taux accélération gcc & 1 & 1,25 & 1,91 & 1,72 & 1,57\\
        \hline
        taux accélération icx & 1 & 1,43 & 1,91 & 1,63 & 1,50\\
        \hline
    \end{tabular}
    \caption{Comparaison entre gcc et icx pour différentes options d'optimisation, version statique}
\end{table}

\begin{table}[H]
    \centering
    \begin{tabular}{|c|c|c|c|c|c|}
        \hline
        \diagbox{compilateur}{Temps total} & -O0 & -O1 & -O2 & -O3 & -Os\\
        \hline
        gcc & 0,469s & 0,380s & 0,377s & 0,247s & 0,413s \\
        \hline
        icx & 0,476s & 0,363s & 0,357s  & 0,239s & 0,369s \\
        \hline
        taux accélération gcc & 1 & 1,24 & 1,24 & 1,90 & 1,14\\
        \hline
        taux accélération icx & 1 & 1,31 & 1,33 & 1,99 & 1,29 \\
        \hline
    \end{tabular}
    \caption{Comparaison entre gcc et icx pour différentes options d'optimisation, version dynamique}
\end{table}
\subsection{Utilisation des options fprofile-generate et fprofile-use}
En faisant la commande \textit{make fprofile}, on compile «dynamique.c» avec l'option \textit{-fprofile-generate} de gcc, on exécute le fichier généré «a.out» ce qui va générer un fichier en .gcda. On compile une nouvelle fois «dynamique.c» avec l'option \textit{-fprofile-use} puis on utilise la commande time pour le fichier généré «a.out».
\newline
On obtient le résultat suivant disponible dans \textit{resultat/res\_fprofile.txt}:
\begin{table}[h!]
    \centering
    \begin{tabular}{|c|c|}
        \hline
        Version & Temps Total \\
        \hline
        sans option & 26,859s \\
        \hline
        avec option &  24,954s\\
        \hline
    \end{tabular}
    \caption{Comparaison entre programme sans options et avec (Exécution 1)}
    \label{tab:graph_characteristics_1}
\end{table}
\begin{table}[h!]
    \centering
    \begin{tabular}{|c|c|}
        \hline
        Version & Temps Total \\
        \hline
        sans options & 24,818s \\
        \hline
        avec options &  26,025s \\
        \hline
    \end{tabular}
    \caption{Comparaison entre programme sans options et avec (Exécution 2)}
    \label{tab:graph_characteristics_1}
\end{table}
On a une amélioration du temps total avec les options \textit{-fprofile-generate} et \textit{-fprofile-use} pour la première exécution. Mais pour la seconde, l'utilisation de ces options n'améliore pas le code. D'une exécution à une autre, l'utilisation des options ne donne pas de bons résultats, ce n'est pas stable.
\clearpage
\part{Man gcc}
Pour cette partie nous utilisons le script «option.sh» pour exécuter la commande
\begin{lstlisting}
    gcc -Q --help=optimizers -O<N> | grep enable | awk '{print $1}'
    avec N ={1, 2, 3, s}
\end{lstlisting}
pour les options O1, O2, O3 et Os. La liste pour chaque option se trouve dans le dossier \textit{resultat/}. Les fichiers sont sous la forme «liste\_option\_ON.txt».

\part{Librairie MKL d'Intel}
Dans cette section nous allons comparer les résultats obtenus avec le compilateur icx avec les options -O3 et -mkl. Ces résultats sont déjà présentés dans \hyperref[sec:mkl]{la section 3.1}. On remarque que les temps totaux avec l'option -mkl sont meilleurs que ceux avec l'option -O3.

\part{Compteurs matériels de performances}
Dans cette section nous allons analyser les événements matériels provoqués par le processus statique\_gcc\_O0.
Avec la commande perf suivante :
\begin{lstlisting}
perf stat -e cycles,cache-misses,instructions,L1-dcache-load-misses,L1-icache-load-misses,branch-instructions,branch-loads,node-loads ./statique_gcc_O0
\end{lstlisting}
On obtient le résultat illustré sur la photo «perfStat.png» disponible dans le dossier \textit{resultat/} où l'on voit le détail des événements matériels enregistrés.

Avec la commande : %«»
\begin{lstlisting}
perf record -e cycles,cache-misses,instructions,L1-dcache-load-misses,L1-icache-load-misses,branch-instructions,branch-loads,node-loads ./statique_gcc_O0
\end{lstlisting}
on obtient un fichier «perf.data\_statique\_O0» qui grâce à la commande
\begin{lstlisting}
perf report perf.data\_statique\_O0
\end{lstlisting}
permet de visualiser également des événements matériels. On peut voir le résultat de cette action sur la photo «perfReport.png» disponible dans le dossier \textit{resultat/}.

\clearpage
\part{Conclusion}
L'évaluation des performances d'un programme est nécessaire pour analyser son temps d'exécution. Nous avons vu au cours de ce TP que nous pouvions améliorer ces performances grâce à des options d'optimisation lors de la compilation de notre programme. L'utilisation d'outils de profilage comme \textit{gprof}, que nous avons utilisés précédemment, est aussi très importante pour analyser en profondeur notre programme. Il existe d'autres outils de profilage spécifiques à certaines architectures, comme \textit{vtune} de chez Intel. Enfin, nous avons également effectué une comparaison entre deux compilateurs pour voir lequel est le plus performant.
\newline
En somme, sans ces analyses, nous ne pouvons pas savoir si notre code est performant ou non.

\end{document}
